\chapter{Signal-Erzeugung}

Die verschiedenen Signalerzeugungungs-Betriebsarten können nur für einen ATmega328-Prozessor
gewählt werden.
Sie müssen außerdem die Dialogfunktion mit der Makefile-Option WITH\_MENU eingeschaltet haben.
Das Bedienmenü kann mit einem langen Tastendruck aufgerufen werden.
Die wählbaren Funktionen werden in der zweiten Zeile des LCD angezeigt.
Die angezeigte Funktion kann man mit einem langen Tastendruck anwählen.
Die nächste Funktion wird automatisch in 5 Sekunden oder nach einem kurzen Tastendruck
angezeigt.

\label{sec:generation}

\section{Frequenz-Erzeugung}
Die Frequenzerzeugung wird gestartet, wenn die Menüfunktion ,,f-Generator'' mit einem
langen Tastendruck angewählt wird.
Die Ausgabe der Frequenz erfolgt über den \(680\Omega\) Widerstand auf den Messport TP2.
Der Messport TP1 wird auf GND geschaltet.
Die Frequenzen werden mit dem 16-Bit Zähler aus der CPU-Taktfrequenz erzeugt (\(8MHz\) oder \(16MHz\)).
Die Frequenz kann dekadenweise beginnend mit der 1Hz Stelle mit den Ziffern 0-9 eingestellt werden.
Die höchste wählbare Stelle ist die 100kHz Stelle. Hier können Zahlen bis zu 20 eingestellt werden.
Somit sind Ausgabefrequenzen bis 2Mhz einstellbar.
Ohne Drehimpulsgeber kann der Stellenwert durch einen kurzen Tastendruck (\textless~0.8s)
erhöht werden.
Mit Drehimpulsgeber können die Stellenwerte beliebig erhöht oder erniedrigt werden.
Mit einem längeren Tastendruck wird die Stelle gewechselt.
Dabei wird in Spalte 1 angezeigt, in welche Richtung die nächste Stelle durch den längeren
Tastendruck gewechselt wird.
Bei einem \textgreater~Zeichen in Spalte 1 der Frequenzzeile wird die nächsthöhere
Stelle gewählt.
Bei einem \textless~Zeichen in Spalte 1  wird die nächstniedrige Stelle gewählt (bis 1Hz).
Wenn die höchste Stelle (100kHz) eingestellt ist, wird anstelle des \textgreater~Zeichen
ein R angezeigt. Dann bewirkt ein längerer Tastendruck ein Rücksetzen der Frequenz auf 1Hz.
Weil mit dem Zähler nicht jede eingestellte Frequenz korrekt erzeugt werden kann,
wird die Frequenzabweichung des erzeugten Signals in Zeile 3 oder hinter dem Frequenzwert gezeigt.
Ein langer Tastendruck (\textgreater~2s) kehrt zu der Dialog-Funktion zurück,
wo die gleiche oder eine andere Funktion gewählt werden kann.

\section{Pulsweiten-Erzeugung}
Die Pulsweiten-Erzeugung wird gestartet, wenn Sie die Funktion ,,10-Bit PWM'' mit einem
langen Tastendruck anwählen.
Die Ausgabe der Frequenz erfolgt über den \(680\Omega\) Widerstand auf den Messport TP2.
Der Messport TP1 wird auf GND geschaltet.
Die Ausgabefrequenz ist fest und ergibt sich aus der CPU-Taktrate dividiert durch 1024.
Das ergibt eine Frequenz von \(7812,5Hz\) für die Taktrate \(8MHz\).
Nur die positive Pulsweite kann mit einem Tastendruck verändert werden. Mit einem
kurzen Tastendruck wird die positive Pulsweite um \(1\%\) bis zu \(99\%\) erhöht.
Mit einem längeren Tastendruck wird die Pulsweite um \(10\%\) erhöht.
Wenn die Pulsweite Werte über \(99\%\) erreicht, wird 100 vom Ergebnis abgezogen.
Die Pulsweite \(0\%\) erzeugt einen sehr kurzen positiven Puls.


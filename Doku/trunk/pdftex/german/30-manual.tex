\chapter{Bedienungshinweise}
\label{sec:manual}
\section{Der Meßbetrieb}
Die Bedienung des Transistortesters ist einfach.
Trotzdem sind einige Hinweise erforderlich.
Meistens sind an die drei Testports über Stecker-Leitungen mit Krokodilklemmen oder anderen Klemmen angeschlossen.
Es können auch Fassungen für Transistoren angeschlossen sein.
In jedem Fall können Sie Bauteile mit drei Anschlüssen mit den drei Testports in beliebiger Reihenfolge verbinden.
Bei zweipoligen Bauteilen können Sie die beiden Anschlüsse mit beliebigen Testports verbinden.
Normalerweise spielt die Polarität keine Rolle, auch Elektrolytkondensatoren können beliebig angeschlossen werden.
Die Messung der Kapazität wird aber so durchgeführt, dass der Minuspol am Testport mit der kleineren Nummer liegt.
Da die Messspannung aber zwischen 0,3 V und maximal 1,3 V liegt, spielt auch hier die Polarität keine wichtige Rolle.
Wenn das Bauteil angeschlossen ist, sollte es während der Messung nicht berührt werden. Legen Sie es auf einen
isolierenden Untergrund ab, wenn es nicht in einem Sockel steckt. Berühren Sie auch nicht die Isolation der Messkabel,
das Messergebnis kann beeinflusst werden.
Dann sollte der Starttaster gedrückt werden.
Nach einer Startmeldung erscheint nach circa zwei Sekunden das Messergebnis. Bei einer Kondensatormessung kann es
abhängig von der Kapazität auch deutlich länger dauern.

Was dann weiter geschieht, hängt von der Softwarekonfiguration des Testers ab.
\begin{description}
  \item[Einzelmessung] Wenn der Tester für Einzelmessung konfiguriert ist, schaltet der Tester nach einer Anzeigezeit
von 28 Sekunden (konfigurierbar) wieder automatisch aus, um die Batterie zu schonen.
Während der Anzeigezeit kann aber auch vorzeitig eine neue Messung gestartet werden.
Nach der Abschaltung kann natürlich auch wieder eine
neue Messung gestartet werden, entweder mit dem gleichen Bauteil oder mit einem anderen Bauteil.
Wenn die Elektronik zum Abschalten fehlt, wird das letzte Meßergebnis weiter angezeigt.

  \item[Dauermessung] Einen Sonderfall stellt die Konfiguration ohne die automatische Abschaltfunktion dar.
Diese Konfiguration wird normalerweise nur ohne die Transistoren für die Abschaltung benutzt.
Es wird stattdessen ein externer Ein-/Aus-Schalter benötigt. Hierbei wiederholt der Tester die
Messungen solange, bis ausgeschaltet wird.

  \item[Serienmessung] In diesem Konfigurationsfall wird der Tester nicht nach einer Messung sondern erst nach einer konfigurierbaren
Zahl von Messungen abgeschaltet. Im Standardfall wird der Tester nach fünf Messungen ohne erkanntes Bauteil
abgeschaltet. Wird ein angeschlossenes Bauteil erkannt, wird erst bei der doppelten Anzahl, also zehn Messungen abgeschaltet.
Eine einzige Messung mit nicht erkanntem Bauteil setzt die Zählung für erkannte Bauteile auf Null zurück.
Ebenso setzt eine einzige Messung mit erkannten Bauteil die Zählung für die nicht erkannten Bauteile auf Null zurück.
Dies hat zur Folge, dass auch ohne Betätigung des Starttasters immer weiter gemessen werden kann,
 wenn Bauteile regelmässig gewechselt werden.
Ein Bauteilwechsel führt in der Regel durch die zwischenzeitlich leeren Klemmen zu einer Messung ohne erkanntes Bauteil.

Eine Besonderheit gibt es in diesem Betriebsmodus für die Anzeigezeit. Wenn beim Einschalten der Starttaster nur kurz
gedrückt wurde, beträgt die Anzeigezeit der Messergebnisse nur 5 Sekunden. Wenn der Starttaster bis zum Erscheinen der
ersten Meldung festgehalten wurde, beträgt die Anzeigezeit wie bei der Einzelmessung 28 Sekunden.
Ein vorzeitiger neuer Messbeginn ist aber während der Anzeigezeit durch erneutes Drücken des Starttasters möglich.

\end{description}

\section{Optionale Menüfunktionen für den ATmega328}
Wenn die Menü Funktion eingeschaltet ist, startet der Tester nach einem längeren Tastendruck (> 300ms) ein Auswahlmenüs
für zusätzliche Funktionen.
Die wählbaren Funktionen erscheinen in Zeile~2 des Displays. Nach 5~Sekunden wechselt die wählbare Funktion automatisch.
Nach Anzeige der letzten Funktion wird wieder die erste angezeigt.
Durch kurzen Tastendruck kann die nächste Auswahl auch schneller gewechselt werden. Mit einem längeren Tastendruck startet
die angezeigte Zusatzfunktion.\\

Bei der Zusatzfunktion f-Generator (Frequenz Generator) werden die wählbaren Frequenzen durch kurzen Tastendruck gewechselt.
Ein längerer Tastendruck kehrt wieder zur Auswahl der Funktionen zurück.\\

Die Zusatzfunktion Frequenz (Frequenzmessung) benutzt den PD4 Pin des ATmega, der auch an das LCD angeschlossen ist.
Es wird immer zunächst die Frequenz gemessen, bei Frequenzen unter \(10 kHz\) wird auch die Periode des Eingangssignals bestimmt
und daraus die Frequenz mit einer Auflösung von bis zu \(0.001 Hz\) berechnet.
Die Frequenzmessung wird durch Tastendruck beendet und in das Auswahlmenü zurückgekehrt.\\

Die Zusatzfunktion Spannung (Spannungsmessung) ist nur möglich, wenn die serielle Ausgabe deaktiviert wurde.
Da am Port PC3 ein 10:1 Spannungsteiler vorgesehen ist, können Spannungen bis 50V gemessen werden.
Die Messung kann auch wieder durch Tastendruck beendet werden.\\

Mit der Zusatzfunktion ,,Schalte aus'' kann der Transistortester abgeschaltet werden.\\

Natürlich kann mit der Funktion Transistor (Transistor Tester) wieder zu der normalen Transistortester Funktion
zurückgekehrt werden. Derzeit sind alle Zusatzfunktionen zeitbeschränkt, damit die Batterie nicht verbraucht wird.

\section{Selbsttest und Kalibration}

Wenn die Software mit der Selbsttestfunktion konfiguriert ist, kann der Selbsttest durch einen Kurzschluss aller drei
Testports und drücken der Starttaste eingeleitet werden.
Für den Start des Selbsttests muß die Starttaste innerhalb von 2 Sekunden noch einmal gedrückt werden,
sonst wird mit einer normalen Messung fortgefahren.
Beim Selbsttest werden die im Selbsttest-Kapitel \ref{sec:selftest} beschriebenen Tests ausgeführt.
Die viermalige Testwiederholung kann vermieden werden, wenn der Starttaster gedrückt gehalten wird.
So kann man uninteressante Tests schnell beenden und
sich durch Loslassen des Starttasters interessante Tests viermal wiederholen lassen.
Der Test 4 läuft nur automatisch weiter, wenn die Verbindung zwischen den Testports gelöst wird.

Wenn die Funktion AUTO\_CAL in der Makefile gewählt ist, wird beim Selbsttest
eine Kalibration der Nullwertes für die Kondensatormessung durchgeführt.
Für die Kalibration des Nullwertes für die Kondensatormessung ist wichtig, 
daß die Verbindung zwischen den Testpins (Kurzschluß) während des Tests 4 wieder gelöst wird!
Sie sollten während der Kalibration (nach dem Test 6) weder die Testports noch angeschlossene Kabel berühren.
Die Ausrüstung sollte aber die gleiche sein, die später zum Messen verwendet wird.
Anderenfalls wird der Nullwert der Kondensatormessung nicht richtig bestimmt.
Die Kalibration des Innenwiderstandes der Port-Ausgänge wird mit dieser Option vor jeder
Messung durchgeführt.

Für den letzten Teil der Kalibration ist der Anschluß eines Kondensators 
mit einer beliebiger Kapazität zwischen \(100 nF\) und \(20 \mu F\) an Pin~1 und Pin~3 erforderlich.
Dazu wird in Zeile 1 ein Kondensatorsymbol zwischen den Pinnummern 1 und 3 angezeigt, gefolgt von dem Text ,, >100nF''.
Sie sollten den Kondensator erst nach dieser Ausgabe anschließen.
Mit diesem Kondensator wird die Offset Spannung des analogen Komparators kompensiert,
um genauere Kapazitätswerte ermitteln zu können.
Die Verstärkung für ADC Messungen mit der internen Referenz-Spannung wird ebenfalls mit diesem Kondensator abgeglichen, um
bessere Widerstands-Messergebnisse mit der AUTOSCALE\_ADC Option zu erreichen.

Der Nullwert für die ESR-Messung wird als Option ESR\_ZERO in der Makefile vorbesetzt.
Mit jedem Selbsttest wird der ESR Nullwert für alle drei Pinkombinationen neu bestimmt.
Das Verfahren der ESR-Messung wird auch für Widerstände mit Werten unter \(10 \Omega\) benutzt um
hier eine Auflösung von \(0.01 \Omega\) zu erreichen.

\section{Besondere Benutzungshinweise}
Normalerweise wird beim Start des Testers die Batteriespannung angezeigt. Wenn die Spannung eine Grenze unterschreitet, 
wird eine Warnung hinter der Batteriespannung ausgegeben. Wenn Sie eine aufladbare 9V Batterie benutzen, sollten Sie
den Akku möglichst bald austauschen oder nachladen.
Wenn Sie eine Version mit eingebauter 2.5V Präzisionsreferenz besitzen, wird beim Start für eine Sekunde die
gemessene Betriebsspannung in der Zeile 2 mit ,,VCC=x.xxV'' angezeigt.

Es kann nicht oft genug erwähnt werden, daß Kondensatoren vor dem Messen entladen sein müssen.
Sonst kann der Tester schon defekt sein, bevor der Startknopf gedrückt ist.
Wenn man versucht, Bauelemente im eingebauten Zustand zu messen, sollte das Gerät immer von
der Stromquelle getrennt sein. Außerdem sollte man sicher sein, daß keine Restspannung mehr
im Gerät vorhanden ist. Alle Geräte haben Kondensatoren verbaut!

Beim Messen kleiner Widerstandswerte muß man besonders auf die Übergangswiderstände achten.
Es spielt die Qualität und der Zustand von Steckverbindern eine große Rolle, genau so wie die
Widerstandwerte von Meßkabeln. Dasselbe gilt auch für die Messung des ESR Wertes von Kondensatoren.
Bei schlechten Anschlußkabeln mit Krokodilklemmen wird so aus einem ESR von \(0.02 \Omega\) leicht
ein Wert von \(0.61 \Omega\).

An die Genauigkeit der Meßwerte sollte man keine übertriebenen Erwartungen haben, dies gilt besonders
für die ESR Messung und die Induktivitätsmessung.
Die Ergebnisse meiner Meßreihen kann man im Kapitel \ref{sec:measurement} finden.


\section{Problemfälle}
Bei den Meßergebnissen sollten Sie immer im Gedächtnis behalten, daß die Schaltung des Transistortesters für
Kleinsignal Bauelemente ausgelegt ist. Normalerweise beträgt der maximale Meßstrom etwa 6 mA.
Leistungshalbleiter machen oft wegen hoher Reststöme Probleme bei der Erkennung oder beim Messen der
Sperrschicht-Kapazität.
Bei Thyristoren und Triacs werden oft die Zündströme oder die Halteströme nicht erreicht. Deswegen kann es
vorkommen, daß ein Thyristor als NPN Transistor oder Diode erkannt wird. Ebenso ist es möglich, daß ein 
Thyristor oder Triac gar nicht erkannt wird.

Probleme bei der Erkennung machen auch Halbleiter mit integrierten Widerständen.
So wird die Basis - Emitter Diode eines BU508D Transistors wegen eines parallel geschalteten internen
\(42 \Omega\) Widerstandes nicht erkannt.
Folglich kann auch die Transistorfunktion nicht geprüft werden.
Probleme bei der Erkennung machen oft auch Darlington Transistoren höherer Leistung. Hier sind auch
oft Basis - Emitter Widerstände verbaut, welche die Erkennung wegen der hier verwendeten kleinen Meßströme erschweren.

\section{Messung von PNP und NPN Transistoren}
Normalerweise werden die drei Anschlüsse des Transistors in beliebiger Reihenfolge an die Meßeingänge des
Transistortesters angeschlossen.
Nach dem Drücken des Starttasters meldet der Tester in der Zeile 1 den Typ (NPN oder PNP), 
eine eventuell vorhandene Schutzdiode der Kollektor - Emitter Strecke und
die Anschlußbelegung. Das Diodensymbol wird  polungsrichtig angezeigt.
In der Zeile 2 wird der Stromverstärkungsfaktor (B=...) und die Basis - Emitter Schwellspannung ausgegeben.
Hierbei sollte man wissen, daß der Tester den Stromverstärkungsfaktor in zwei Schaltungsvarianten 
ermittelt, der Emitterschaltung und der Kollektorschaltung (Emitterfolger).

Bei der Emitterschaltung hat der Tester nur zwei Möglichkeiten, den Basisstrom einzustellen:
\begin{enumerate}
\item Mit dem \(680 \Omega\) Widerstand ergibt sich ein Basisstrom von etwa 6.1mA . Das ist für einen
Kleinsignaltransistor mit hohem Verstärkungsfaktor meist zu viel, weil die Basis gesättigt ist.
Da der Kollektorstrom ebenfalls mit einem \(680 \Omega\) Widerstand gemessen wird, kann der Kollektorstrom
den um den Verstärkungsfaktor höheren Strom gar nicht erreichen. Die Softwareversion vom Markus F. hat
in diesem Zustand die Basis - Emitter Spannung gemessen (Uf=...).\\
\item Mit dem \(470 k\Omega\) Widerstand] ergibt sich ein Basisstrom von nur \(9.2 \mu A\) .
Das ist für einen Leistungstransistor mit geringem Verstärkungsfaktor sehr wenig.
Die Softwareversion von Markus F. hat in diesem Zustand den Stromverstärkungsfaktor bestimmt (hFE=...).\\
\end{enumerate}

Die Software des Testers bestimmt die Stromverstärkung jetzt auch in der Kollektorschaltung.
Ausgegeben wird der höhere Wert der beiden Mess-Methoden.
Die Kollektorschaltung hat den Vorteil, daß sich durch die Stromgegenkopplung der Basisstrom abhängig vom
Verstärkungsfaktor reduziert. Dadurch kann für Leistungstransistoren mit dem \(680 \Omega\) und für Darlington Transistoren
mit dem \(470 k\Omega\) Widerstand meist ein günstigerer Meßstrom ergeben.
Die ausgegebene Basis Emitter Schwellspannung Uf ist jetzt die Spannung,
die bei der Bestimmung des Stromverstärkungsfaktors gemessen wurde.
Wenn man trotzdem eine Basis - Emitter Schwellspannung bei circa 6 mA ermitteln möchte, muß man den Kollektor
vom Tester trennen und noch einmal messen.
Dann wird die Schwellspannung bei etwa 6mA ausgegeben und die Kapazität der Diode in Sperr-Richtung ermittelt.
Natürlich kann so auch die Basis - Kollektor Diode gemessen werden.

Bei Germanium Transistoren wird meistens ein Kolleḱtor-Emitter Reststrom \(I_{CE0}\) mit stromloser Basis oder
ein Kollektor-Emitter Reststrom \(I_{CES}\) mit Basis auf Emitterpotential gemessen.
In diesem Fall werden in Zeile 2 für ATmega328 für 5 Sekunden oder bis zum Tastendruck die Restströme
vor der Stromverstärkung ausgegeben.
Durch Kühlen kann der Reststrom bei Germanium Transistoren erheblich gesenkt werden. 

\section{Messung von JFET und D-MOS Transistoren}
Wegen des symmetrischen Aufbaus von JFET Transistoren kann Source und Drain nicht unterschieden werden.
Normalerweise wird bei JFET als eine Kenngröße der Strom bei kurzgeschlossenem Gate - Source angegeben.
Dieser Strom ist aber oft höher als der, der sich bei der Meßschaltung mit dem \(680 \Omega\) Widerstand erreichen läßt.
Deswegen wird der \(680 \Omega\) Widerstand an den Source Anschluß geschaltet. Dadurch erhält das Gate
stromabhängig eine negative Vorspannung. Als Kenngröße wird sowohl der ermittelte Strom als auch die
Gate - Source Spannung ausgegeben. 
Damit können verschiedene Typen unterschieden werden.
Für D-MOS Transistoren (Verarmungs-Typ) wird das gleiche Meßverfahren verwendet.

Für Anreicherungs MOS Transistoren (P-E-MOS oder N-E-MOS) sollte man wissen, daß die Bestimmung der Gate Schwellspannung (Vth)
bei kleiner Gate  Kapazität schwierig wird. Hier können mit dem Tester genauere Spannungswerte ermittelt werden, wenn ein
Kondensator mit einigen nF parallel zum Gate / Source angeschlossen wird.
Die Schwellspannung wird bei Drain Strömen von etwa 3.6mA für P-E-MOS und bei etwa 4mA für N-E-MOS bestimmt.

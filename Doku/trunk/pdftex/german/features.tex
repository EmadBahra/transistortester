%\newpage
\chapter{Eigenschaften}
\label{sec:features}
\begin{enumerate}
\item Arbeitet mit ATmega8, ATmega88, ATmega168 oder ATmega328 Mikrocontrollern.
\item Anzeige der Me"sergebnisse auf einem 2x16 Zeichen LCD-Display.
\item Ein Tasten Bedienung mit automatischer Abschaltfunktion.
\item Batterie-Betrieb ist m\"oglich, weil der Strom im abgeschalteten Zustand nur etwa 20nA betr\"agt.
\item Preisg\"unstige L\"osung ist m\"oglich ohne Quarz und ohne automatische Abschaltung.
\item Automatische Erkennung von NPN und PNP bipolaren Transistoren, N- und P-Channel MOSFETs, JFETs,
Dioden, Doppeldioden, Thyristors und Triacs.
\item Automatische Erkennung der Pin Anschl\"usse der erkannten Bauteile.
\item Messung des Stromverst\"arkungsfaktors und der Basis-Emitter Schwellspannung f\"ur bipolare Transistoren.
\item Darlington Transistoren k\"onnen durch die h\"ohere Schwellspannung und durch den hohen Stromverst\"arkungsfaktor erkannt werden.
\item Automatische Erkennung einer Schutzdiode bei bipolaren Transistoren und bei MOSFETs.
\item Messung der Schwellwert-Spannung und der Gate Kapazit\"atswerte von MOSFETs.
\item Bis zu zwei Widerst\"ande werden gemessen und mit Symbolen
\setlength{\unitlength}{0.1mm}
\linethickness{0.4mm}
\begin{picture}(60,30)
\put(0,15){\line(1,0){10}}
\put(10,5){\line(0,1){20}}
\put(10,5){\line(1,0){40}}
\put(10,25){\line(1,0){40}}
\put(50,5){\line(0,1){20}}
\put(50,15){\line(1,0){10}}
\end{picture}
und den Widerstands-Werten angezeigt.
Alle Symbole werden eingerahmt mit den gefundenen Testpin Nummern des Testers (1-3).
Deshalb k\"onnen auch Potentiometer gemessen werden. Wenn der Schleifer eines Potentiometers auf eine Endposition
gestellt ist, kann der Tester nicht mehr zwischen mittlerem Anschlu"s und Endanschlu"s unterscheiden.
\item Die Aufl\"osung der Widerstandsmessung ist jetzt \(0,1\Omega\), Werte von bis zu \(50M\Omega\) werden erkannt.
\item Ein Kondensator kann erkannt und gemessen werden. Der wird mit dem Symbol 
\setlength{\unitlength}{0.1mm}
\begin{picture}(60,30)
\linethickness{0.4mm}
\put(0,15){\line(1,0){20}}
\put(40,15){\line(1,0){20}}
\put(22,0){\line(0,1){30}}
\put(26,0){\line(0,1){30}}
\put(34,0){\line(0,1){30}}
\put(38,0){\line(0,1){30}}
\end{picture}
und Kapazit\"atswert angezeigt.
Der Wert kann zwischen 25pF (bei 8MHz Takt, 50pF bei 1MHz Takt) bis 40mF mit einer Aufl\"osung von bis zu 1pF (bei 8MHz Takt).
\item Bis zu vier Dezimalstellen werden f\"ur die Kapazit\"ats und Widerstands-Werte in der richtigen Dimension angezeigt.
\item Bis zu zwei Dioden werden mit dem Symbol
\setlength{\unitlength}{0.1mm}
\begin{picture}(60,30)
\linethickness{0.4mm}
\put(0,15){\line(1,0){60}}
\put(22,2){\line(0,1){26}}
\put(26,6){\line(0,1){18}}
\put(30,10){\line(0,1){10}}
\put(38,2){\line(0,1){26}}
\end{picture}
oder dem Symbol
\setlength{\unitlength}{0.1mm}
\begin{picture}(60,30)
\linethickness{0.4mm}
\put(0,15){\line(1,0){60}}
\put(38,2){\line(0,1){26}}
\put(34,6){\line(0,1){18}}
\put(30,10){\line(0,1){10}}
\put(22,2){\line(0,1){26}}
\end{picture}
in der richtigen Reihenfolge angezeigt.
Zus\"atzlich werden die Schwellspannungen angezeigt.
\item Eine LED wird als Diode erkannt, die Schwellspannung ist viel h\"oher als bei einer normalen Diode.
Doppeldioden werden als zwei Dioden erkannt.
\item Zener-Dioden k\"onnen erkannt werden, wenn die Zener-Spannung unter 4,5V ist.
Sie werden als zwei Dioden angezeigt, man kann das Bauelement nur mit den Spannungen erkennen.
Die \"au"seren Testpin Nummern, welche die Dioden Symbole umgeben, sind in diesem Fall identisch.
Man kann die wirkliche Anode der Diode nur durch diejenige Diode herausfinden, deren Schwellwert Spannung nahe bei 700mV liegt!
\item Wenn mehr als 3 Dioden erkannt werden, wird die gefundene Anzahl der Dioden zusammen mit der Fehlermeldung angezeigt.
Das kann nur passieren, wenn Dioden an alle drei Test-Pins angeschlossen sind und wenigstens eine eine Zener-Diode ist.
In diesem Fall sollte man nur zwei Test-Pins anschlie"sen und die Messung erneut starten, eine nach der anderen.
\item Der Kapazit\"atswert einer einzelnen Diode in Sperr-Richtung wird automatisch ermittelt.
Bipolare Transistoren k\"onnen auch untersucht werden, wenn nur die Basis und entweder Kollektor oder Emitter angeschlossen wird.
\item Die Anschl\"usse einer Gleichrichter-Br\"ucke k\"onnen mit nur einer Messung herausgefunden werden.
\item Kondensatoren mit Kapazit\"atswerten von unter 25pF werden normalerweise nicht erkannt, 
aber sie k\"onnen zusammen mit einer parallel geschalteten Diode oder mit einem parallel geschaltetem Kondensator mit
wenigstens 25pF gemessen werden.
In diesem Fall mu"s der Kapazit\"atswert des parallel geschalteten Bauteils vom Me"sergebnis abgezogen werden.
\item Die Me"szeit betr\"agt ungef\"ahr zwei Sekunden, nur Kapazit\"atsmessungen k\"onnen l\"anger dauern.
\item Die Software kann f\"ur Me"sserien mit vorgebbarer Wiederhol-Zahl konfiguriert werden, bevor die automatische Abschaltung ausschaltet.
\item Eingebaute Selbsttest-Funktion inklusive einem 50Hz Frequenz-Generator um die Genauigkeit der Taktfrequenz und der Verz\"ogerungszeiten zu \"uberpr\"ufen.
\item W\"ahlbare M\"oglichkeit, den Nullabgleich f\"ur die Kondensatormessung und die Innenwiderstandsbestimmung f\"ur die
Portausg\"ange beim Selbsttest abzugleichen.
\end{enumerate}

Thyristoren und Triacs k\"onnen nur erkannt werden, wenn der Test-Strom \"uber dem Halte-Strom liegt.
Einige Thyristoren und Triacs brauchen auch einen h\"oheren Z\"undstrom als dieser Tester liefern kann.
Der verf\"ugbare Teststrom ist nur ungef\"ahr 6mA!
Es ist m\"oglich, da"s nicht alle Optionen in zuk\"unftigen Versionen erhalten bleiben, weil diese Software
immer noch im Teststadium ist.

\vspace{1cm}
\textbf{{\Large Achtung:}} Stellen Sie immer sicher, da"s Kondensatoren vor dem Anschlu"s an den Tester entladen sind!
Der Tester k\"onnte sonst besch\"adigt werden bevor er eingeschaltet ist.
Es gibt nur wenig Schutzfunktion der ATmega Anschl\"usse.



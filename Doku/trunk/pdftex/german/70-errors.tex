
%\newpage
\chapter{Bekannte Fehler und ungelöste Probleme}
{\center Software-Version 1.11k}

\begin{enumerate}

\item Germanium-Dioden (AC128) werden nicht in allen Fällen entdeckt. Ursache ist vermutlich der Reststrom.
Das Kühlen der Diode reduziert den Reststrom.

\item Der Stromverstärkungsfaktor von Germanium Transistoren kann zu hoch gemessen werden wegen dem hohen Reststrom.
In diesem Fall ist die gemessene Basis Emitter Spannung auffällig klein.
Das Kühlen des Transistors kann helfen, einen realistischeren Stromverstärkungsfaktor zu bestimmen.

\item Bei Leistungs-Doppeldioden vom Schottky-Typ wie MBR3045PT kann bei Anschluß einer Einzeldiode keine Kapazitätsmessung in Sperr-Richtung 
durchgeführt werden. Der Grund ist ein zu hoher Reststrom. Der Fehler kann manchmal durch Kühlen (Kältespray) umgangen werden.

\item Es ist gelegentlich zu einer Falscherkennung einer 2.5V Präzisionsreferenz gekommen, wenn der Anschluß PC4 (Pin 27) unbeschaltet ist.
Abhilfe ist möglich mit einem zusätzlichen Pull Up Widerstand nach VCC.

\item Die Dioden Funktion des Gates eines Triac kann nicht untersucht werden.

\item Vereinzelt ist von Problemen mit der Brown Out Schwelle 4.3V für ATmega168 oder ATmega328 Prozessoren berichtet worden.
Dabei kommt es zu Resets bei der Kondensatormessung.  Eine Ursache ist nicht bekannt.
Der Fehler verschwindet, wenn man die Brown Out Schwelle auf 2.7V einstellt.

\item Mit der Benutzung des Schlafzustandes des processors schwankt die VCC Stromaufnahme mehr als
in älteren Software Versionen.
Sie sollten die Abblock-Kondensatoren überprüfen, wenn Sie irgendwelche Probleme feststellen.
Keramische 100nF Kondensatoren sollten in der Nähe der Poweranschlüsse des ATmega angeschlossen sein.
Sie können die Benutzung des Schlafzustandes auch mit der Makefile Option INHIBIT\_SLEEP\_MODE verhindern.

\item Das Messen von Tantal Elektrolyt-Kondensatoren macht oft Probleme.
Sie können als Diode erkannt werden oder auch gar nicht erkannt werden.
Manchmal hilft ein Polaritätswechsel.

\item Die Anschlüsse Source und Drain können bei JFET's nicht richtig ermittelt werden.
Die Ursache liegt in dem symmetrischen Aufbau dieser Halbleiter.
Erkennen kann man dieses Problem, daß die Anzeige mit den ermittelten Parametern gleich bleibt,
wenn die Anschlüsse vertauscht angeschlossen werden.
Leider kenne ich keinen Weg, Source und Drain korrekt zu ermitteln.
Aber das Vertauschen von Source und Drain in irgendeiner Schaltung sollte normalerweise kein Problem verursachen.

\end{enumerate}

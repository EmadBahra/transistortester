
%\newpage
\chapter{Bekannte Fehler und ungelöste Probleme}
{\center Software-Version 1.04k}

\begin{enumerate}

\item Germanium-Dioden (AC128) werden nicht in allen Fällen entdeckt. Ursache ist vermutlich der Reststrom.
\item Bei Leistungs-Doppeldioden vom Schottky-Typ wie MBR3045PT kann bei Anschlu"s einer Einzeldiode keine Kapazitätsmessung in Sperr-Richtung 
durchgeführt werden. Der Grund ist ein zu hoher Reststrom. Der Fehler kann manchmal durch Kühlen (Kältespray) umgangen werden.
\item Es ist gelegentlich zu einer Falscherkennung einer 2.5V Präzisionsreferenz gekommen, wenn der Anschluß PC4 (Pin 27) unbeschaltet ist.
Abhilfe ist möglich mit einem zusätzlichen Pull Up Widerstand nach VCC.
\item Die Dioden Funktion des Gates eines Triac kann nicht untersucht werden.
\item Vereinzelt ist von Problemen mit der Brown Out Schwelle 4.3V für ATmega168 oder ATmega328 Prozessoren berichtet worden.
Dabei kommt es zu Resets bei der Kondensatormessung.  Eine Ursache ist nicht bekannt.
Der Fehler verschwindet, wenn man die Brown Out Schwelle auf 2.7V einstellt. 

\end{enumerate}

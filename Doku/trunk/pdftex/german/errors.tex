
%\newpage
\chapter{Bekannte Fehler und ungel\"oste Probleme}
{\center Software Version 0.97k}

\begin{enumerate}
\item Die Me"sergebnisse f\"ur kleine Kondensatoren sind abh\"angig von der gew\"ahlten Pin Kombination.
Die Ergebnisse der Kombination 1:2 sind 3pF kleiner als die beiden anderen Kombinationen (1:3 und 2:3).
Dieser Effekt ist unabh\"angig vom verwendeten AVR Prozessor Typ.
\item Arbeitet das Programm richtig ohne die automatische Abschalt-Funktion?
\item Die angezeigte Pin Nummern bei der Widerstandsmessung sind manchmal gleich und verbleiben in diesem
Fehlerzustand bis die Versorgungsspannung abgeschaltet wird.
(Version 0.95k).
\item Kondensatoren mit Kapazit\"aten von mehr als 40mF werden als Widerstand mit einem Wert von \(2\Omega\) erkannt.
Weil Widerst\"ande zuerst entdeckt werden, wird die Kapazit\"atsmessung dann nicht mehr durchgef\"uhrt.
Praxisgerechtere Kondensatoren haben meist Werte von bis zu 10mF.
Deshalb k\"onnte die Kapazit\"atsmessung auch auf 40mF anstelle der geplanten 100mF begrenzt werden, wenn dieses Problem nicht
gel\"ost werden kann.

\end{enumerate}

\chapter{Signal-Erzeugung}

Die verschiedenen Signalerzeugungungs-Betriebsarten können nur für einen ATmega328-Prozessor
gewählt werden.
Sie müssen außerdem die Dialogfunktion mit der Makefile-Option WITH\_MENU eingeschaltet haben.
Das Bedienmenü kann mit einem langen Tastendruck aufgerufen werden.
Die wählbaren Funktionen werden in der zweiten Zeile des LCD angezeigt.
Die angezeigte Funktion kann man mit einem langen Tastendruck anwählen.
Die nächste Funktion wird automatisch in 5 Sekunden oder nach einem kurzen Tastendruck
angezeigt.

\label{sec:generation}

\section{Frequenz-Erzeugung}
Die Frequenzerzeugung wird gestartet, wenn die Menüfunktion ,,f-Generator'' mit einem
langen Tastendruck angewählt wird.
Die Ausgabe der Frequenz erfolgt über den \(680\Omega\) Widerstand auf den Messport TP2.
Der Messport TP1 wird auf GND geschaltet.
Die Frequenzen werden mit dem 16-Bit Zähler aus der CPU-Taktfrequenz erzeugt (8 MHz oder 16 MHz).
Zur Zeit kann eine vordefinierte Liste von Frequenzen (2 MHz runter bis zu 1 Hz) mit einem kurzen
Tastendruck gewählt werden. Ein langer Tastendruck kehrt zu der Dialog-Funktion zurück,
wo die gleiche oder eine andere Funktion gewählt werden kann.

\section{Pulsweiten-Erzeugung}
Die Pulsweiten-Erzeugung wird gestartet, wenn Sie die Funktion ,,10-Bit PWM'' mit einem
langen Tastendruck anwählen.
Die Ausgabe der Frequenz erfolgt über den \(680\Omega\) Widerstand auf den Messport TP2.
Der Messport TP1 wird auf GND geschaltet.
Die Ausgabefrequenz ist fest und ergibt sich aus der CPU-Taktrate dividiert durch 1024.
Das ergibt eine Frequenz von \(7812.5 Hz\) für die Taktrate 8 MHz.
Nur die positive Pulsweite kann mit einem Tastendruck verändert werden. Mit einem
kurzen Tastendruck wird die positive Pulsweite um 1\% bis zu 99\% erhöht.
Mit einem längeren Tastendruck wird die Pulsweite um 10\% erhöht.
Wenn die Pulsweite Werte über 99\% erreicht, wird 100 vom Ergebnis abgezogen.
Die Pulsweite 0\% erzeugt einen sehr kurzen positiven Puls.


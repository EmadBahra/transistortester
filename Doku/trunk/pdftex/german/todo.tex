
%\newpage
\chapter{Arbeitsliste und neue Ideen}
\label{sec:todo}

\begin{enumerate}
\item Erg\"anze mehr und bessere Dokumentation.
\item Pr\"ufe, ob der TransistorTester besser ,,interpolierte''  ADC Werte erhalten kann, wenn zus\"atz\-liches Rauschen zum Signal
oder zur Referenz hinzugef\"ugt wird (siehe ATMEL Dokument AVR121 \cite{AVR121}: Enhancing ADC resolution by oversampling).
Wenn alle Werte gleich sind, kann keine Verbesserung der Aufl\"osung durch \"Uberabtasten stattfinden.
Kann gen\"ugend Rauschen mit den ATmega Z\"ahlern erzeugt werden?
Nat\"urlich kann diese Methode nicht alle ADC Fehler beseitigen.
\item Diese Methode kann getestet werden, indem ein langsam ansteigendes Eingangssignal erzeugt wird und dieses
Signal beobachtet wird.
Das ansteigende Signal kann erzeugt werden indem ein gro"ser Kondensator mit dem \(470k\Omega\) Widerstand geladen wird.
Das Anwachsen der Spannung kann dann mit dem LCD-Display in einem speziellen Teil des Selbst-Testes beobachtet werden.
Die Unterschiede der ReadADC Alternativen \(\frac{44}{9}, \frac{22\cdot2}{9}\) oder \(\frac{11\cdot4}{9}\) k\"onnen
ebenfalls beobachtet werden.
\item Der ADC arbeitet mit einer Taktrate von 125kHz. Die Spezifikation erlaubt einen Betrieb mit 200kHz bei voller Genauigkeit.
Aber 200kHz Taktfrequenz kann mit dem Vorteiler nicht erzeugt werden, wenn die CPU-Taktrate 1MHz oder 8MHz betr\"agt.
Wie viel Genauigkeit geht verloren, wenn die ADC Taktrate auf 250kHz gesetzt wird?
Die Messungen k\"onnten in etwa der halben Zeit erledigt werden, wenn der 250kHz Betrieb toleriert werden k\"onnte.
Bisherige Tests waren nicht ermutigend.
\item Dar\"uber nachdenken, wie man den wirklichen Innenwiderstand der Port B Aus\-g\"an\-ge (Wider\-stands-Schal\-ter) bestimmen kann,
anstatt anzunehmen, da"s die Ports gleich sind.
\item Strom Idss von Drain to Source messen (oder Innenwiderstand?) mit Gate auf Source Potential for JFET.
\item Kann das Entladen von Kondensatoren beschleunigt werden, wenn der Minus-Pol zus\"atzlich mit dem \(680\Omega\) 
Widerstand nach VCC (+) geschaltet wird?
\item K\"onnen Induktivit\"aten gemessen werden?
\item Wie ver\"andern sich die Me"sergebnisse, wenn die Betriebs-Spannung zwischen 4,5V und 5V variiert wird?
\item Pr\"ufe, ob der Tester Flie"skomma Darstellung von Werten gebrauchen kann.
Das \"Uberlaufsrisiko (overflow) ist geringer.
Man braucht keinen Multiplikation / Division Konstruktion um einen Faktor mit einer gebrochenen Zahl nachzubilden.
Aber ich wei"s nicht wieviel Platz f\"ur die Bibliothek gebraucht wird.
\item Schreibe eine Gebrauchsanweisung zum Konfigurieren des Testers mit den Makefile Optionen und beschreibe
den Ablauf bis zum fertigen Prozessor.
\item Wenn der Haltestrom eines Thyristors nicht mit dem \(680\Omega\) Widerstand erreicht werden kann, 
ist es ungef\"ahrlich f\"ur eine sehr kurze Zeit die Kathode direkt auf GND und die Anode direkt auf VCC zu schalten?
Der Strom kann mehr als 100mA erreichen. Wird der Port besch\"adigt? Was ist mit der Spannungsversorgung (Spannungsregler)?
\item Pr\"ufe die Ports nach dieser Aktion mit der Selbsttest-Funktion!
\item K\"onnen Spannungsregler gepr\"uft werden? (Eingang, Ausgang, GND)
\item K\"onnen Optokoppler getestet werden?
\item Ist es m\"oglich den ESR von Elektrolyt-Kondensatoren zu bestimmen? Zum Beispiel beim Entladen mit den Port C Ausg\"angen und beim
Laden mit dem \(680\Omega\) Widerstand.
\item Ausgabe einer Warnung, wenn die relativ zu VCC gemessene Referenz-Spannung nicht plausibel im Hinblick auf das ATmega Modell ist.
\item Kann mit einer an PC4 angeschlossenen Pr\"azisions-Spannungsreferenz die VCC Spannung und die interne Referenz kalibriert werden?
\item Wie w\"are es mit einem neuen Tester mit einem gr\"o"seren ATmega, der differenzielle ADC-Ports hat,
mehr Programm (flash) Speicher ...?
Es gibt keinen ATxmega, der eine Versorgungs-Spannung von 5V hat, nur die ATmega Linie ist m\"oglich.
\item Idee f\"ur ein neues Projekt: USB  Version ohne LCD-Display, Power vom USB Port, Kommunikation zum PC \"uber eine USB-Serial Br\"ucke.
\end{enumerate}

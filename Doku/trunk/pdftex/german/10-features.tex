%\newpage
\chapter{Eigenschaften}
\label{sec:features}
\begin{enumerate}
\item Arbeitet mit Mikrocontrollern vom Typ ATmega8, ATmega168 oder ATmega328.
\item Anzeige der Messergebnisse auf einem 2x16 Zeichen großen LC-Display.
\item Ein-Tasten-Bedienung mit automatischer Abschaltfunktion.
\item Batterie-Betrieb ist möglich, weil der Strom im abgeschalteten Zustand nur etwa 20nA beträgt.
\item Preisgünstige Lösung ist möglich ohne Quarz und ohne automatische Abschaltung.
\item Automatische Erkennung von NPN und PNP bipolaren Transistoren, N- und P-Channel MOSFETs, JFETs,
Dioden, Doppeldioden, Thyristoren und Triacs.
\item Darstellung der Pin-Belegung der erkannten Bauteile.
\item Messung des Stromverstärkungsfaktors und der Basis-Emitter-Schwellspannung für bipolare Transistoren.
\item Darlington-Transistoren können durch die höhere Schwellspannung und durch den hohen Stromverstärkungsfaktor erkannt werden.
\item Automatische Erkennung einer Schutzdiode bei bipolaren Transistoren und bei MOSFETs.
\item Messung der Schwellwert-Spannungen und der Gate-Kapazitätswerte von MOSFETs.
\item Bis zu zwei Widerstände werden gemessen und mit Symbolen
\setlength{\unitlength}{0.1mm}
\linethickness{0.4mm}
\begin{picture}(60,30)
\put(0,15){\line(1,0){10}}
\put(10,5){\line(0,1){20}}
\put(10,5){\line(1,0){40}}
\put(10,25){\line(1,0){40}}
\put(50,5){\line(0,1){20}}
\put(50,15){\line(1,0){10}}
\end{picture}
und den Widerstands-Werten mit bis zu vier Dezimalstellen in der richtigen Dimension angezeigt.
Alle Symbole werden eingerahmt mit den gefundenen Testpin Nummern des Testers (1-3).
Deshalb können auch Potentiometer gemessen werden. Wenn der Schleifer eines Potentiometers auf eine Endposition
gestellt ist, kann der Tester nicht mehr zwischen mittlerem Anschluss und Endanschluss unterscheiden.
\item Die Auflösung der Widerstandsmessung ist jetzt \(0,1\Omega\), Werte von bis zu \(50M\Omega\) werden erkannt.
\item Ein Kondensator kann erkannt und gemessen werden. Der wird mit dem Symbol 
\setlength{\unitlength}{0.1mm}
\begin{picture}(60,30)
\linethickness{0.4mm}
\put(0,15){\line(1,0){20}}
\put(40,15){\line(1,0){20}}
\put(22,0){\line(0,1){30}}
\put(26,0){\line(0,1){30}}
\put(34,0){\line(0,1){30}}
\put(38,0){\line(0,1){30}}
\end{picture}
und Kapazitätswert mit bis zu vier Dezimalstellen in der richtigen Dimension angezeigt.
Der Wert kann zwischen 25pF (bei 8MHz Takt, 50pF bei 1MHz Takt) bis 100mF liegen. Die Auflösung kann bis zu 1pF (bei 8MHz Takt) betragen.
\item Bei Kondensatoren mit einer Kapazität über \(2 \mu F\) wird zusätzlich der äquivalente Serienwiderstand (ESR) des Kondensators
mit einer Auflösung von \(0.01 \Omega\) gemessen und mit zwei Dezimalstellen angezeigt.
Diese Fähigkeit steht nur zur Verfügung, wenn der ATmega mindestens 16K Flashspeicher besitzt.
\item Bis zu zwei Dioden werden mit dem Symbol
\setlength{\unitlength}{0.1mm}
\begin{picture}(60,30)
\linethickness{0.4mm}
\put(0,15){\line(1,0){60}}
\put(22,2){\line(0,1){26}}
\put(26,6){\line(0,1){18}}
\put(30,10){\line(0,1){10}}
\put(38,2){\line(0,1){26}}
\end{picture}
oder dem Symbol
\setlength{\unitlength}{0.1mm}
\begin{picture}(60,30)
\linethickness{0.4mm}
\put(0,15){\line(1,0){60}}
\put(38,2){\line(0,1){26}}
\put(34,6){\line(0,1){18}}
\put(30,10){\line(0,1){10}}
\put(22,2){\line(0,1){26}}
\end{picture}
in der richtigen Reihenfolge angezeigt.
Zusätzlich werden die Schwellspannungen angezeigt.
\item Eine LED wird als Diode erkannt, die Schwellspannung ist viel höher als bei einer normalen Diode.
Doppeldioden werden als zwei Dioden erkannt.
\item Zener-Dioden können erkannt werden, wenn die Zener-Spannung unter 4,5V ist.
Sie werden als zwei Dioden angezeigt, man kann das Bauelement nur mit den Spannungen erkennen.
Die äußeren Testpin-Nummern, welche die Dioden Symbole umgeben, sind in diesem Fall identisch.
Man kann die wirkliche Anode der Diode nur durch diejenige Diode herausfinden, deren Schwellwert-Spannung nahe bei 700mV liegt!
\item Wenn mehr als 3 Dioden erkannt werden, wird die gefundene Anzahl der Dioden zusammen mit der Fehlermeldung angezeigt.
Das kann nur passieren, wenn Dioden an alle drei Test-Pins angeschlossen sind und wenigstens eine eine Zener-Diode ist.
In diesem Fall sollte man nur zwei Test-Pins anschließen und die Messung erneut starten, eine Diode nach der anderen.
\item Der Kapazitätswert einer einzelnen Diode in Sperr-Richtung wird automatisch ermittelt.
Bipolare Transistoren können auch untersucht werden, wenn nur die Basis und entweder Kollektor oder Emitter angeschlossen wird.
\item Die Anschlüsse einer Gleichrichter-Brücke können mit nur einer Messung herausgefunden werden.
\item Kondensatoren mit Kapazitätswerten von unter 25pF werden normalerweise nicht erkannt, 
aber sie können zusammen mit einer parallel geschalteten Diode oder mit einem parallel geschaltetem Kondensator mit
wenigstens 25pF gemessen werden.
In diesem Fall muß der Kapazitätswert des parallel geschalteten Bauteils vom Meßergebnis abgezogen werden.
\item Bei Widerständen unter \(2100~\Omega\) wird auch eine Induktivitätsmessung durchgeführt, wenn der
ATmega mehr als 8K Flashspeicher besitzt (ATmega168 oder ATmega328).
Der Anzeigebereich ist etwa \(0.01~mH\) bis über \(20~H\), die Genauigkeit ist allerdings nicht hoch.
Das Ergebnis wird nur bei einem Einzelwiderstand zusammen mit dem Widerstandswert angezeigt.
\item Die Messzeit beträgt ungefähr zwei Sekunden, nur Kapazitätsmessungen und Induktivitätsmessungen können länger dauern.
\item Die Software kann für Messserien mit vorgebbarer Wiederhol-Zahl konfiguriert werden, bevor die automatische Abschaltung ausschaltet.
\item Eingebaute Selbsttest-Funktion inklusive einem optionalen 50Hz Frequenz-Generator um die Genauigkeit der Taktfrequenz
und der Verzögerungszeiten zu überprüfen (nur mit ATmega168 oder ATmega328).
\item Wählbare Möglichkeit den Nullabgleich für die Kondensatormessung und die Innenwiderstände für die
Portausgänge beim Selbsttest automatisch zu bestimmen (nur mit ATmega168 oder ATmega328).
Ein externer Kondensator mit einer Kapazität zwischen \(100~nF\) und \(20~\mu F\) an Pin~1 und Pin~3 ist notwendig, 
um die Offset Spannung des analogens Komparators zu kompensieren.
Dies kann den Messfehler bei Kapazitätsmessungen bis zu \(40~\mu F\) reduzieren.
Mit dem gleichen Kondensator wird eine Korrekturspannung zum Einstellen der richtigen Verstärkung für
die ADC Messung mit der internen \(1.1V\) Referenzspannung berechnet.
\end{enumerate}

Thyristoren und Triacs können nur erkannt werden, wenn der Test-Strom über dem Halte-Strom liegt.
Einige Thyristoren und Triacs brauchen auch einen höheren Zündstrom als dieser Tester liefern kann.
Der verfügbare Teststrom ist nur ungefähr 6mA!
Beachten Sie bitte, dass alle Möglichkeiten nur für Mikrocontroller mit genug Programmspeicher zur Verfügung stehen (ATmega168 oder ATmega328).

\vspace{1cm}
\textbf{{\Large Achtung:}} Stellen Sie immer sicher, dass {\bf Kondensatoren} vor dem Anschluss an den Tester {\bf entladen} sind!
Der Tester könnte sonst beschädigt werden bevor er eingeschaltet ist.
Es gibt nur wenig Schutzfunktion der ATmega Anschlüsse.
Besondere Vorsicht ist auch geboten, wenn versucht wird, Bauelemente in einer Schaltung zu messen.
Das Gerät sollte in jedem Fall vorher von der Strom\-ein\-spei\-sung getrennt sein und man sollte sicher sein,
daß {\bf keine Restspannung} im Gerät vorhanden ist.



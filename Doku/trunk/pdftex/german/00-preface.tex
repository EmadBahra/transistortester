\section*{Vorwort}

\subsection*{Grundsätzliches}
Jeder Bastler kennt das folgende Problem: Man baut einen Transistor aus oder man nimmt einen aus einer Bastelkiste.
Wenn man die Typenbezeichnung erkennen kann und man bereits ein Datenblatt hat oder eins bekommen kann, ist alles in Ordnung.
Aber wenn man keine Datenblätter findet, hat man keine Idee, was das für ein Bauteil sein kann.
Mit den konventionellen Meßmethoden ist es schwierig und zeitaufwändig den Typ des Bauteils und dessen Parameter herauszufinden.
Es könnte ein NPN, PNP, N- oder P-Kanal-MOSFET usw. sein. 
Es war die Idee von Markus F., diese Arbeit von einem AVR-Mikrocontroller erledigen zu lassen.

\subsection*{Wie meine Arbeit begann}
Meine Beschäftigung mit der Software des TransistorTesters von Markus F. \cite{Frejek} hat begonnen, als ich Probleme mit
meinem Programmer hatte.
Ich hatte eine Platine und Komponenten gekauft, aber ich war mit dem Windows-Treiber nicht in der Lage den EEprom-Speicher des ATmega8
ohne Fehlermeldungen zu beschreiben.
Deshalb habe ich die Software von Markus F. genommen und habe alle Zugriffe auf den EEprom-Speicher durch
Zugriffe auf den Programm-Speicher (Flash) ersetzt.
Bei der Durchsicht der Software, um an anderer Stelle Programmspeicher (Flash) einzusparen, hatte ich die Idee,
das Ergebnis der ReadADC Funktion von ADC-Einheiten in eine Millivolt (mV) Auflösung zu ändern.
Die mV Auflösung wird für die Ausgabe von Spannungswerten gebraucht.
Wenn die ReadADC-Funktion direkt die mV Auflösung liefert, kann man die Umwandlung für jeden
Ausgabewert einsparen.
Diese mV-Auflösung kann man erhalten, wenn man zuerst die Ergebnisse von 22 ADC-Einlesungen addiert.
Die Summe muß mit zwei multipliziert und durch neun geteilt werden.
Das ergibt einen Maximalwert von \begin{math}\frac{1023\cdot22\cdot2}{9} = 5001\end{math},
welcher hervorragend zu der gewünschten mV Auflösung passt.
So hatte ich zusätzlich die Hoffnung, dass die Erhöhung der ADC-Auflösung durch Überabtastung helfen
könnte, die Spannungs-Einlesung zu verbessern, wie es in dem Atmel Report AVR121 \cite{AVR121} beschrieben ist.
Die Original-Version von ReadADC hat die Ergebnisse von 20 ADC-Einlesungen addiert und danach durch 20 dividiert,
so dass das Ergebnis wieder die Original-Auflösung des ADC hat. Deshalb konnte niemals eine Erhöhung der ADC-Auflösung
durch Überabtastung stattfinden.
So hatte ich wenig Arbeit, die ReadADC Funktion zu ändern, aber dies erforderte die Analyse des kompletten
Programms und Änderung aller ,,if''-Abfragen im Programm, wo Spannungswerte geprüft wurden.
Aber dies war nur der Beginn meiner Arbeit!

Mehr und mehr Ideen wurden eingebaut, um die Messung schneller und genauer zu machen.
Zusätzlich wurde der Bereich der Widerstands- und Kondensator-Messung erweitert.
Das Ausgabe-Format für das LC-Display wurde verändert, so wurden Symbole für die Darstellung von
Dioden, Widerständen und Kondensatoren verwendet.
Für weitere Einzelheiten schauen Sie in das aktuelle Eigenschaften-Kapitel \ref{sec:features}.
Geplante Arbeiten und neue Ideen wurden im Arbeitslisten-Kapitel \ref{sec:todo} gesammelt.
Inzwischen kann ich unter dem Linux-Betriebssystem auch den EEprom-Speicher des ATmega8 einwandfrei beschreiben.


\section*{Vorwort}

\subsection*{Grunds\"atzliches}
Jeder Bastler kennt das folgende Problem: Man baut einen Transistor aus oder man nimmt einen aus einer Bastelkiste.
Wenn man die Typenbezeichnung erkennen kann und man bereits ein Datenblatt hat oder eins bekommen kann, ist alles in Ordnung.
Aber wenn man keine Datenbl\"atter findet, hat man keine Idee, was das f\"ur ein Bauteil sein kann.
Mit den konventionellen Me"smethoden ist es schwierig und zeitaufw\"andig den Typ des Bauteils und dessen Parameter herauszufinden.
Es k\"onnte ein NPN, PNP, N- oder P-Kanal-MOSFET usw. sein. 
Es war die Idee von Markus F., diese Arbeit von einem AVR-Mikrocontroller erledigen zu lassen.

\subsection*{Wie meine Arbeit begann}
Meine Besch\"aftigung mit der Software des TransistorTesters von Markus F. \cite{Frejek} hat begonnen, als ich Probleme mit
meinem Programmer hatte.
Ich hatte eine Platine und Komponenten gekauft, aber ich war mit dem Windows-Treiber nicht in der Lage den EEprom-Speicher des ATmega8
ohne Fehlermeldungen zu beschreiben.
Deshalb habe ich die Software von Markus F. genommen und habe alle Zugriffe auf den EEprom-Speicher durch
Zugriffe auf den Programm-Speicher (Flash) ersetzt.
Bei der Durchsicht der Software, um an anderer Stelle Programmspeicher (Flash) einzusparen, hatte ich die Idee,
das Ergebnis der ReadADC Funktion von ADC-Einheiten in eine Millivolt (mV) Aufl\"osung zu \"andern.
Die mV Aufl\"osung wird f\"ur die Ausgabe von Spannungswerten gebraucht.
Wenn die ReadADC-Funktion direkt die mV Aufl\"osung liefert, kann man die Umwandlung f\"ur jeden
Ausgabewert einsparen.
Diese mV-Aufl\"osung kann man erhalten, wenn man zuerst die Ergebnisse von 22 ADC-Einlesungen addiert.
Die Summe mu"s mit zwei multipliziert und durch neun geteilt werden.
Das ergibt einen Maximalwert von \begin{math}\frac{1023\cdot22\cdot2}{9} = 5001\end{math},
welcher hervorragend zu der gew\"unschten mV Aufl\"osung passt.
So hatte ich zus\"atzlich die Hoffnung, dass die Erh\"ohung der ADC-Aufl\"osung durch \"Uberabtastung helfen
k\"onnte, die Spannungs-Einlesung zu verbessern, wie es in dem Atmel Report AVR121 \cite{AVR121} beschrieben ist.
Die Original-Version von ReadADC hat die Ergebnisse von 20 ADC-Einlesungen addiert und danach durch 20 dividiert,
so dass das Ergebnis wieder die Original-Aufl\"osung des ADC hat. Deshalb konnte niemals eine Erh\"ohung der ADC-Aufl\"osung
durch \"Uberabtastung stattfinden.
So hatte ich wenig Arbeit, die ReadADC Funktion zu \"andern, aber dies erforderte die Analyse des kompletten
Programms und \"Anderung aller ,,if''-Abfragen im Programm, wo Spannungswerte gepr\"uft wurden.
Aber dies war nur der Beginn meiner Arbeit!

Mehr und mehr Ideen wurden eingebaut, um die Messung schneller und genauer zu machen.
Zus\"atzlich wurde der Bereich der Widerstands- und Kondensator-Messung erweitert.
Das Ausgabe-Format f\"ur das LC-Display wurde ver\"andert, so wurden Symbole f\"ur die Darstellung von
Dioden, Widerst\"anden und Kondensatoren verwendet.
F\"ur weitere Einzelheiten schauen Sie in das aktuelle Eigenschaften-Kapitel \ref{sec:features}.
Geplante Arbeiten und neue Ideen wurden im Arbeitslisten-Kapitel \ref{sec:todo} gesammelt.
Inzwischen kann ich unter dem Linux-Betriebssystem auch den EEprom-Speicher des ATmega8 einwandfrei beschreiben.


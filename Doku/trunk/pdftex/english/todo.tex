
%\newpage
\chapter{To Do List and new ideas}
\label{sec:todo}

\begin{enumerate}
\item Add more and better documentation.
\item By my tests I have noticed that the measured voltages of the internal band gap reference is lower than the data sheets let me expect. The reason is unknown. VCC?, ADC-error?
\item Check if transistor tester could get better “interpolated” ADC values if additional noise is added to the signal or to the ADC reference (see ATMEL document AVR121: Enhancing ADC resolution by oversampling).
If all items are identical, there can't be any enhancement of resolution by oversampling. Can enough noise be generated with the ATmega counter?
How additional noise affects  the upper and lower limit values?
Of course this method can not eliminate all of the ADC errors. 
\item This method can be tested by building a ramp input signal and monitoring this signal.
The ramp signal can be build by slowly charging a big capacitor with the 470kΩ resistor.
The growing of the voltage can then be monitored with the LCD display in a special part of self test.
\item The ADC operates with a clock frequency of 125kHz. The specification allows up to 200kHz with full accuracy.
But the 200kHz clock is impossible to set with prescaler, if CPU-Clock is 1MHz or 8MHz.
How much accuracy is lost, if the ADC Clock is set to 250kHz?
Measurements could be done in nearly half the time, if 250kHz operation is tolerable.
\item Think about how we can get the real internal resistance of port B output (resistor switching port) instead of assuming, that ports are equal.
\item Can discharging of capacitors be made more quickly, if the minus pin is additionally raised
with the \(680\Omega\) resistor to VCC (+)?
\item Who is using the serial port? I did not test this function and even I don't know how.
\item Can inductance be tested?
\item How measurement results changes by variation of  the  supply voltage between 4,5V and 5V? 
\item Check if the tester can use floating-point representation of  values. The risk of overflow  is lower.
There is no need to use multiplication and division together to build a multiplication with a non integer factor.
But I don't know how much flash memory must be spend for the library.
\item Write User's guide for configuring the tester with the Makefile options and description of the build chain.
\item If the holding current of a thyristor can not be reached with the \(680\Omega\) resistor, is it harmless to switch the cathode directly to GND
and the anode directly to VCC for a very short time?
The current could reach more than 100mA. Will the port be damaged? What is with the power supply (voltage regulator)?
\item Check the Port afterwards with self test function!
\item Can voltage regulators be checked? (Input, Output, GND)
\item Can optoelectronic couplers be checked?  
\item Is the ESR measurement of electrolytical capacitors  possible.
\item Warning message, if the found reference voltage is not plausible in relation to ATmega model and VCC.
\item Can we connect precision voltage reference to PC4 to calibrate VCC and internal reference?
\item If a battery cell is connected to the tester, the tester tries to discharge,
but fails without message, better is it to recognise the part as cell (with voltage?).
\item What is about a second generation tester with a bigger ATmega which includes differential ADC-port, more flash memory …. ?
There is no ATxmega which have supply voltage of 5V, only the ATmega line  is possible. 
\item Idea for a New Projekt: USB  version without LCD-Display, Power from USB, Communication to PC over a USB-Serial bridge.
\end{enumerate}

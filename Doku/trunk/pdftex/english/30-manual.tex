\chapter{Instructions for use}
\label{sec:manual}
\section{The measurement operation}
Using of the Transistor-Tester is simple.
Anyway some hints are required.
In most cases are wires with alligator clips connected to the test ports with plugs.
Also sockets for transistors can be connected.
In either case you can connect parts with three pins to the three test ports in any order.
If your part has only two pins, you can connect this pins to any two of the tree test ports.
Normally the polarity of part is irrelevant, you can also connect pins of electrolytical capacitors in any order. 
The measurement of capacity is normally done in a way, that the minus pole is at the test port with the lower number.
But, because the measurment voltage is only between 0.3 V and at most 1.3 V, the polarity doesn\'t matter.
When the part is connected, you should not touch it during the measurement. You should put it down to a nonconducting pad
if it is not placed in a socket. You should also not touch to the isolation of wires connected with the test ports,
the measurement results can be affected.
Then you should press the start button.
After displaying a start message, the measurement result should appear after two seconds.
If capacitors are measured, the time to result can be longer corresponding to the capacity.

How the transistor-tester continues, depends on the configuration of the software.
\begin{description}
  \item[Single measurement mode] If the tester is configured for single measurement mode, the tester shut off automatical after displaying the
result for 10 seconds for a longer lifetime of battery. 
During the display time a next measurement can be started by pressing the start button.
After the shut off a next measurement can be started too of course.
The next measurement can be done with the same or another part.
If you have not installed the electronic for automatic shut down, your
last measurement result will be displayed until you start the next measurement.

  \item[Endless measurement mode] A special case is the configuration without automatical shut off.
This configuration is normally only used without the transistors for the shut off function.
A external off switch is necessary for this case. The tester will repeat measurements until power
is switched off.

  \item[Multi measurement mode] In this mode the tester will shut down not after the first measurement but 
after a configurable series of measurements. In the standard case the tester will shut down after five
measurements without found part. If any part is identified by test, the tester is shut down after double of
five (ten) measurements. A single measurement with unknown part after a series of measurement of known parts will
reset the counter of known measuerements to zero. Also a single measurement of known part will reset the counter
of unknown measurements to zero. This behavior can result in a nearly endless series of measurements without
pressing the start button, if parts are disconnected and connected in periodical manner.

In this mode there is a special feature for the display period. If the start button is pressed only short for switching
on the tester, the result of measurement ist only shown for 5 seconds. Buf if you press and hold the start button until
the first message is shown, the further measurement results are shown for 14 seconds.
The next measurement can started earlier by pressing the start button during the displaying of result.

\end{description}

\section{Selftest and Calibration}

If the software is configured with the selftest function, the selftest can be started by connecting all three
test ports together and pushing of the start button.
In this case all of the documented tests in the Selftest chapter \ref{sec:selftest} will be done.
The repetition of the tests can be avoided, if the start button is hold pressed.
So you can skip uninteresting tests fast and you can watch interresting tests by releasing the start button.
The test 4 will finish only automatically if you separate the test ports (release connection).

If the function AUTO\_CAL is selected in the Makefile, 
the zero offset for the capacity measurement will be calibrated with the selftest.
It is important for the calibration task, that the connection between the three test ports is relased 
during test number 4. 
You should not touch to any of the test ports or connected cables when calibration (after test 6) is done.
But the equipment should be the same used for further measurements. Otherwise the zero offset for
capacity measurement is not detected correctly.
The resistance values of port outputs are determined at the beginning of every measurement with this option.

A capacitor with any capacity between \(100 nF\) and \(20 \mu F\) connected to pin~1 and pin~3 is
required for the last task of calibration.
To indicate that, a capacitor symbol is shown between the pin number 1 and 3, followed by the text '' \textgreater 100nF''.
You should connect the capacitor not before this text is shown.
With this capacitor the offset voltage of the analog comparator will be compensated for better measurement
of capacity values.  
Additionally the gain for ADC measurements using the internal reference voltage will be adjusted too 
with the same capacitor for better resistor measurement results with the AUTOSCALE\_ADC option.

The zero offset for the ESR measurement will be preset with the option ESR\_ZERO in the Makefile.
With every self test the ESR zero values for all three pin combinations are determined.
The solution for the ESR measurement is also used to get the values of resistors below \(10 \Omega\) with
a resolution of \(0.01 \Omega\).


\section{special using hints}
Normally the Tester shows the battery voltage with every start. If the voltage fall below a limit,
a warning is shown behind the battery voltage. If you use a rechargeable 9V battery, you should replace
the battery as soon as possible or you should recharge.
If you use a tester with attached 2.5V precision reference, the measured supply voltage will be shown
in display row two for 1 second with ''VCC=x.xxV''.

It can not repeat often enough, that capacitors should be discharged before measuring.
Otherwise the Tester can be damaged before the start button is pressed.
If you try to measure components in assembled condition, the equipment should be allways disconnected from power source.
Furthermore you should be shure, that no residual voltage reside in the equipment.
Every electronical equipment has capacitors inside!

If you try to measure little resistor values, you should keep the resistance of plug connectors and cables in mind.
The quality and condition of plug connectors are important, also the resistance of cables used for measurement.
The same is in force for the ESR measurement of capacitors.
With poor connection cable a ESR value of \(0.02 \Omega\) can grow to \(0.61 \Omega\).

You should not expect very good accuracy of measurement results, especially the ESR measurement and the results of inductance measurement are not very exact.
You can find the results of my test series in chapter \ref{sec:measurement}.

\section{Compoments with problems}
You should keep in mind by interpreting the measurement results, that the circuit of the TransistorTester is
designed for small signal semiconductors. In normal measurement condition the measurement current can only reach about 6 mA.
Power semiconductors often make trouble by reason of residual current with the identification an the measurement of junction capacity value.
The Tester often can not deliver enough ignition current or holding current for power Thyristors or Triacs.
So a Thyristor can be detected as NPN transistor or diode. Also it is possible, that a Thyristor or Triac is detected as unknown.

Another problem is the identification of semiconductors with integrated resistors.
So the base - emitter diode of a BU508D transistor can not be detected by reason of the parallel connected
internal \(42 \Omega\) resistor.
Therefore the transistor function can not be tested also.
Problem with detection is also given with power Darlington transistors. We can find often internal
base - emitter resistors, which make it difficult to identify the component with the undersized measurement current.

\section{Measurement of PNP and NPN transistors}
For normal measurement the three pins of the transistor will be connectet in any order to the measurement
inputs of the TransistorTester.
After pushing the start button, the Tester shows in row 1 the type (NPN or PNP), 
a possible integrated protecting diode of the Collector - Emitter path and the
sequence of pins. The diode symbol is shown with correct polarity.
Row 2 shows the current amplification factor (B=...) and the Base - Emitter threshold voltage.
You shouls know, that the Tester can measure the amplification factor with two different circuits,
the common Emitter and the common Collector circuit (Emitter follower).


With the common Emitter circuit the tester has only two alternative to select the base current:
\begin{enumerate}
\item The \(680 \Omega\) resistor results to a base current of about 6.1mA. 
This is too high for low level transistors with high amplification factor, because the base is saturated.
Because the collector current is also measured with a \(680 \Omega\) resistor, the collector current
can not reach the with the amplification factor higher value.
The software version of Markus F. has measured the Base - Emitter threshold voltage in this ciruit (Uf=...).\\
\item The \(470 k\Omega\) resistor results to a base current of only \(9.2 \mu A\) .
This is very low for a power transistor with low current amplification factor.
The software version of Markus F. has identified the current amplification factor with this circuit (hFE=...).\\
\end{enumerate}

The software of the Tester figure out the current amplification factor additionally with the common Collector circuit.
The higher value of both measurement methodes is reported.
The common collector circuit has the advantage, that the base current is reduced by negative current feedback corresponding
to the amplification factor. 
In most cases a better measurement current can be reached with this methode for power transistors
with the \(680 \Omega\) resistor and for Darlington Transistors with \(470 k\Omega\) resistor.
The reported Base - Emitter threshold voltage Uf is now measured with the same current used 
for determination of the current amplification factor.
However, if you want to know the Base - Emitter threshold voltage with a measurement current of about 6mA,
you have to disconnect the Collector and to start a new measurement.
With this connection, the Base - Emitter threshold voltage at 6 mA is reported. The capacity value
in reverse direction of the diode is also reported.
Of course you can also analyse the base - collector diode.

\section{Measurement of JFET and D-MOS transistors}
Because the structure of JFET type is symmetrical, the Source and Drain of this transistores can not
be differed.
Normally one of the parameter of this transistor is the current of the transistor with the Gate at the same level as Source.
This current is often higher than the current, which can be reached with the measurement circuit of the TransistorTester
with the \(680 \Omega\) resistor.
For this reason the \(680 \Omega\) resistor is connected to the Source. Thus the Gate get with the growing of current a negative
bias voltage.
The Tester reports the Source current of this circuit and additionally the bias voltage of the Gate.
So various models can be differed.
The D-MOS transistors (depletion type) are measured with the same methode.

You should know for enhancement MOS transistors (P-E-MOS or N-E-MOS), that the measurement of the gate threshold voltage (Vth)
is more difficult with little gate capacity values. You can get a better voltage value, if you connect a capacitor with a value
of some nF parallel to the gate /source.
The gate threshold voltage will be find out with a drain current of about 3.5mA for a P-E-MOS and about 4mA for a N-E-MOS.


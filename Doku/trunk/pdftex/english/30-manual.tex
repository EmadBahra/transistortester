\chapter{Instructions for use}
\label{sec:manual}
Using of the Transistor-Tester is simple.
Anyway some hints are required.
In most cases are wires with alligator clips connected to the test ports with plugs.
Also sockets for transistors can be connected.
In either case you can connect parts with three pins to the three test ports in any order.
If your part has only two pins, you can connect this pins to any two of the tree test ports.
Normally the polarity of part is irrelevant, you can also connect pins of electrolytical capacitors in any order. 
The measurement of capacity is normally done in a way, that the minus pole is at the test port with the lower number.
But, because the measurment voltage is only between 0.3 V and at most 1.3 V, the polarity doesn\'t matter.
When the part is connected, you should not touch it during the measurement. You should put it down to a nonconducting pad
if it is not placed in a socket. You should also not touch to the isolation of wires connected with the test ports,
the measurement results can be affected.
Then you should press the start button.
After displaying a start message, the measurement result should appear after two seconds.
If capacitors are measured, the time to result can be longer corresponding to the capacity.

How the transistor-tester continues, depends on the configuration of the software.
\begin{description}
  \item[Single measurement mode] If the tester is configured for single measurement mode, the tester shut off automatical after displaying the
result for 10 seconds for a longer lifetime of battery. 
During the display time a next measurement can be started by pressing the start button.
After the shut off a next measurement can be started too of course.
The next measurement can be done with the same or another part.\\

  \item[Endless measurement mode] A special case is the configuration without automatical shut off.
This configuration is normally only used without the transistors for the shut off function.
A external off switch is necessary for this case. The tester will repeat measurements until power
is switched off.\\

  \item[Multi measurement mode] In this mode the tester will shut down not after the first measurement but 
after a configurable series of measurements. In the standard case the tester will shut down after five
measurements without found part. If any part is identified by test, the tester is shut down after double of
five (ten) measurements. A single measurement with unknown part after a series of measurement of known parts will
reset the counter of known measuerements to zero. Also a single measurement of known part will reset the counter
of unknown measurements to zero. This behavior can result in a nearly endless series of measurements without
pressing the start button, if parts are disconnected and connected in periodical manner.

In this mode there is a special feature for the display period. If the start button is pressed only short for switching
on the tester, the result of measurement ist only shown for three seconds. Buf if you press and hold the start button until
the first message is shown, the further measurement results are shown for ten seconds.
The next measurement can started earlier by pressing the start button during the displaying of result.\\

\end{description}

If the software is configured with the selftest function, the selftest can be started by connecting all three test ports together.
In this case all of the documented tests in the Selftest chapter \ref{sec:selftest} will be done.
The repetition of the tests can be avoided, if the start button is hold pressed.
So you can skip uninteresting tests fast and you can watch interresting tests by releasing the start button.
If the function AUTO\_CAL is selected in the Makefile, there will be calibrated 
the internal resistance of the port output and 
the zero offset for the capacity measurement.
If you connect a capacitor with a capacity value between \(100 nF\) and \(20 \mu F\) to pin~1 and pin~3 after the measurement of the capacity zero offset, 
the offset voltage of the analog comparator will be compensated for better measurement of capacity values and 
additionally the gain for ADC measurements with the internal reference voltage will be adjusted too 
for better resistor measurement results with the AUTOSCALE\_ADC option.


The zero offset for the ESR measurement will be preset as option ESR\_ZERO in the Makefile.
This zero offset, which is set too high in normal case, will be preset in EEprom by the software and will be
set to this initial value by every selftest.
After every ESR measurement the result will be checked for negative value. In this case the zero offset will
be reduced to get a zero result for next ESR measurement.
With this methode the zero offset can be adjusted with a electrolytical capacitor with high capacity value and
low ESR value.
But this adjust precedure must be repeated after every selftest.

%\newpage
\chapter{Features}
\label{sec:features}
\begin{enumerate}
\item Operates with ATmega8, ATmega168 or ATmega328 microcontrollers.
\item Displaying the results to a 2x16 character LCD-Display.
\item One key operation with automatic power shutdown.
\item Battery operation is possible since shutdown current is only about 20nA.
\item Low cost version is feasible without crystal and auto power off.
With software version 1.05k the sleep modus of the Atmega168 or ATmega328 is used to reduce current if
no measurement is required.
\item Automatic detection of NPN and PNP bipolar transistors, N- and P-Channel MOSFETs, JFETs,
diodes, double diodes, Thyristors and Triacs.
\item Automatic detection of pin layout of the detected part.
\item Measuring of current amplification factor and Base-Emitter threshold voltage of bipolar transistors.
\item Darlington transistors can be identified by the threshold voltage and high current amplification factor.
\item Detection of the protection diode of bipolar transistors and MOSFETs.
\item Measuring of the Gate threshold voltage and Gate capacity value of MOSFETs.
\item Up to two Resistors are measured and shown with symbols
\setlength{\unitlength}{0.1mm}
\linethickness{0.4mm}
\begin{picture}(60,30)
\put(0,15){\line(1,0){10}}
\put(10,5){\line(0,1){20}}
\put(10,5){\line(1,0){40}}
\put(10,25){\line(1,0){40}}
\put(50,5){\line(0,1){20}}
\put(50,15){\line(1,0){10}}
\end{picture}
and values with up to four decimal digits in the right dimension.
All symbols are surrounded by the probe numbers of the Tester (1-3).
So Potentiometer can also be measured. If the Potentiometer is adjusted to one of its ends,
the Tester cannot differ the middle pin and the end pin.
\item Resolution of resistor measurement is now up to \(0.01\Omega\), values up to \(50M\Omega\) are detected.
\item One capacitor can be detected and measured. It is shown with symbol
\setlength{\unitlength}{0.1mm}
\begin{picture}(60,30)
\linethickness{0.4mm}
\put(0,15){\line(1,0){20}}
\put(40,15){\line(1,0){20}}
\put(22,0){\line(0,1){30}}
\put(26,0){\line(0,1){30}}
\put(34,0){\line(0,1){30}}
\put(38,0){\line(0,1){30}}
\end{picture}
and value with up to four decimal digits in the right dimension. 
The value can be from 25pF (8MHz clock, 50pF @1MHz clock) to 100mF. The resolution can be up to 1 pF (@8MHz clock].
\item For capacitors with a capacity value above \(0.18 \mu F\) the Equivalent Serial Resistance (ESR) is measured 
with a resolution of \(0.01 \Omega\) and shown with two significant decimal digits.
This feature is only avaiable for ATmega with at least 16K flash memory (ATmega168 or ATmega328).
\item For capacitors with a capacity value above \(5000 pF\) the voltage loss after a load pulse can be determined.
The voltage loss give a hint for the quality factor of the capacitor.
\item Up to two diodes are shown with symbol
\setlength{\unitlength}{0.1mm}
\begin{picture}(60,30)
\linethickness{0.4mm}
\put(0,15){\line(1,0){60}}
\put(22,2){\line(0,1){26}}
\put(26,6){\line(0,1){18}}
\put(30,10){\line(0,1){10}}
\put(38,2){\line(0,1){26}}
\end{picture}
or symbol
\setlength{\unitlength}{0.1mm}
\begin{picture}(60,30)
\linethickness{0.4mm}
\put(0,15){\line(1,0){60}}
\put(38,2){\line(0,1){26}}
\put(34,6){\line(0,1){18}}
\put(30,10){\line(0,1){10}}
\put(22,2){\line(0,1){26}}
\end{picture}
in correct order. Additionally the flux voltages are shown.
\item LED is detected as diode, the flux voltage is much higher than normal. 
Two-in-one LEDs are also detected as two diodes.
\item Zener-Diodes can be detected, if reverse break down Voltage is below 4.5V.
These are shown as two diodes, you can identify this part only by the voltages.
The outer probe numbers, which surround the diode symbols, are identical in this case.
You can identify the real Anode of the diode only by the one with break down (threshold) Voltage nearby 700mV!
\item If more than 3 diode type parts are detected, the number of founded diodes is shown additionally to the fail message.
 This can only happen, if Diodes are attached to all three probes and at least one is a Z-Diode.
In this case you should only connect two probes and start measurement again, one after the other.
\item Measurement of the capacity value of a single diode in reverse direction.
Bipolar Transistors can also be analysed, if you connect the Base and only one of Collector or Emitter.
\item Only one measurement is needed to find out the connections of a bridge rectifier.
\item Capacitors with value below 25pF are usually not detectet, but can be measured together with
a parallel diode or a parallel capacitor with at least 25pF.
In this case you must subtract the capacity value of the parallel connected part.
\item For resistors below \(2100 \Omega\) also the measurement of inductance will be done, if
your ATmega has at least 16K flash memory.
The range will be from about \(0.01 mH\) to more than \(20 H\), but the accuracy is not good.
The measurement result is only shown with a single component connected.
\item Testing time is about two seconds, only capacity or inductance measurement can cause longer period.
\item Software can be configured to enable series of measurements before power will be shut down.
\item Build in selftest function with optional 50Hz Frequency generator to check the accuracy of clock frequency and wait calls (ATmega168 and ATmega328 only).
\item Selectable facility to calibrate the internal port resistance of port output and
the zero offset of capacity measurement with the selftest (ATmega168 and ATmega328 only).
A external capacitor with a value between \(100 nF\) 
and \(20 \mu F\) connected to pin~1 and pin~3 is necessary to compensate the offset voltage of the analog comparator.
This can reduce measurement errors of capacitors of up to \(40 \mu F\).
With the same capacitor a correction voltage to the internal reference voltage is found to adjust the
gain for ADC measuring with the internal reference.
\item Display the Collector cutoff current \(I_{CE0}\) with currentless base (\(10\mu A\) units) and
Collector residual current \(I_{CES}\) with base hold to emitter level (ATmega328 only).
This values are only shown, if they are not zero (especially for Germanium transistors).
\end{enumerate}

Thyristors and Triacs can only be detected, if the test current is above the holding current.
Some Thyristors and Triacs need as higher gate trigger current, than this Tester can deliver.
The available testing current is only about 6mA!
Notice that all features can only be used with microcontroller with more program memory such as ATmega168.

\vspace{1cm}
\textbf{{\Large Attention:}} Allways be shure to {\bf discharge capacitors} before connecting them to the Tester!
The Tester may be damaged before you have switched it on. There is only a little protection at the ATmega ports.

Extra causion is required if you try to test components mounted in a circuit.
In either case the equipment should be disconnected from power source and you should be shure,
that {\bf no residual voltage} remains in the equipment.

